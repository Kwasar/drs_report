\section{Discussion}
\label{sec:conclusion}
The study done in this report used a lot of simplifications and therefore has a huge potential for improvement.
We manly showed optimizations on a simple cantilever micro-switch design and we saw that we quickly reached a limit.
The next step in the development of a resistive switch would therefore be, to think about new mechanical designs, which would help in optimizing the switch.

In terms of power efficiency a bistable flexible structure could be used, allowing the switch only to consume power during the actuation.
Such a design would require some rethinking of the actuation, since a simple electrostatic actuator only works in one direction.
An approach that uses a piezoelectric actuator or two electrostatic actuators could be imagined.
Trade-offs would certainly need to be done between the stiffness of the switch and the force needed to actuate it.

As we have seen earlier in the report, response time of the switch and the force used to actuate the switch are related.
To augment the force, an increase of the actuation area could be imagined.
Since this area is most efficient towards the tip of the cantilever, a cross-like structure could be used.
This would allow to keep lever arm long while augmenting the actuation area and therefore the force.
A disadvantage would be the augmentation of the mass of the beam and therefore the decrease in resonance frequency and response time.
A trade-off would need to be done to explore the efficiency and parameter choices of such a design.

Our study was always done by imagining the environment to be the open air.
Since the device would need to be packaged anyway, it would make sense to create an artificial atmosphere inside this package, which could change the dielectric properties by a lot.
Gases like $SF_6$ or $N_2$ represent a higher breakdown voltage than air and would therefore be a good alternative, that would need to be examined.
Such a gas would also reduce sparking during contact and would therefore augment the lifetime of the contact pads.

As mentioned in the introduction, lifetime of mechanical switches is the biggest challenge in their design.
There are many effects that damage the structure or the contact pads of the switch.
"Cold"\footnote{Actuation of the switch without applying power during actuation.} switching would allow to test for mechanical failures like structural fatigue, memory effect and stiction of the actuator.
"Hot"\footnote{Actuation of the switch while under load.} switching, which is the most common use of a switch, is also the more restricting one in term of lifetime.
Malfunction is mostly caused due to electric failure mechanisms, like temperature, current density and material transfer.\cite{shaw2012mems}
Especially the latter effects require more profound studies to maximize the lifetime of mechanical micro-switches.

\subsection{Conclusion}
\label{sec:conclusion}
Seeing new applications of MEMS was always fascinating to us.
By studying, designing and trading off a mechanical micro-switch, we were able to experience the complexity behind such systems.
This project allowed us to understand, that the development takes a lot of design iterations and smart trade-offs with a huge amount of parameters.
Modeling and simulation is very important to refine a design, but a good understanding of the theory and working principles is necessary to get good results.
