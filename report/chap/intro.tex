\section{Introduction}
\label{sec:introduction}
Today, field-effect transistors (FETs) and PIN diodes are the dominant components used as micro-switches. 
They are reliable, well understood, and good models of them exist. 
The wide variety of applications for these switches induces a need of better performances; the main parameters being low power consumption, high isolation properties, low insertion loss\footnote{The loss of signal power caused by it entering a device.}, a fast response time, and a small footprint.
We nedd to explore new technologies to further improve the performance of micro-switches.

Resistive switches, also known as mechanical relays,  present such a new technology.
Their main advantages in comparison to FETs and PIN diodes lies in their lower power consumption, higher isolation properties, and their wide operation frequency band, which reaches from DC to tens of GHz.
Their working principle is based on performing a mechanical movement to connect or disconnect contact pads to allow or stop current flow.
The challenges lie in designing a flexible structure that guides the contacts to connect, to find an actuation principle that is both fast and low in power consumption, and to chose a contact material that enables good electrical contact as well as high resistance against it's mechanical use.

This report focuses on the study of a simple design to allow trade-offs that are mainly focused on a fast response time, a minimal actuation voltage, and a minimal footprint.
The mechanical design is based on a cantilever.
It's design will be explained in more detail in chapter~\ref{sec:resisitive_switch_design}.\cite{shaw2012mems}
