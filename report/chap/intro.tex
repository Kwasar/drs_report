\section{Introduction}
\label{sec:introduction}
Today, field-effect transistors and PIN diodes are the dominant components used as micro-switches. 
They are very reliable and are very well understood and modelled. 
The vide variety of applications for these switches induces a need of better performances; the main parameters being low power consumption, high isolation properties, low insertion loss\footnote{The loss of a power signal caused by entering a device.}, a fast response time and a small footprint.
New technology needs to be explored to further improve the performance of micro-switches.

One such technology is represented by resistive switches, also called mechanical relays.
Their main advantages lying in lower power consumption, higher isolation properties and a wide operation frequency band, from DC to tens of GHz, compared to conventional micro-switches.
Their working principle is based on performing a mechanical movement to connect or disconnect contact pads to allow or stop current flow.
The challenges lie in designing a flexible structure that guides the contacts to connect, to find an actuation principle that is both fast and low in power consumption, and to chose a contact material that enables good electrical contact and high resistance against it's mechanical use.

This report focuses on the study of a very simple design to allow trade-offs that are mainly focused on a fast response time, a minimum actuation voltage and a minimum footprint.
The mechanical design is based on a cantilever.
It's design will be explained in more detail in the following chapter (chap. \ref{sec:resisitive_switch_design}).\cite{shaw2012mems}
