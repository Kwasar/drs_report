\section*{Abstract}
\label{sec:abstract}

A directional radiation sensor was developed for the JUICE (JUpiter ICy moons Explorer) mission from ESA. 
This mission shall investigate the emergence of habitable worlds around gas giants. 
Its instrumentation shall permit new studies on Jupiter's atmosphere and magnetosphere.\cite{JUICE}
\newline
It is a radiation detector with a shielding that allows each event on its array of 36 detector diodes to be associated with the direction and the type of the particle that was observed. 
The reason why the detector shall be employed on the CubeSat GTO mission is the following; the ability to record not only the characteristics of the generated pulse on the semiconductor detectors but also the incident direction will be imperative in gaining and accurate representation and characterization of the radiation environment in the Van Allen belts.
\newline
At the end of the semester project the scientists and engineers at PSI were in the process of testing the performance of an Application Specific Integrated Circuit (ASIC), which receives the analog pulses from the directional sensor and processes them based on the mode in which it is set up to operate. 
The goal of this project is to design the analog hardware, which connects the sensor, the ASIC, an FPGA, and other supporting components.
Furthermore the digital hardware on an FPGA chip that interacts with said ASIC and reads out the processed information stored in the ASIC digital subsystem shall be designed and tested. 
The low-level circuit diagrams as well as an interface between the ASIC and a computer shall be established and it is demonstrated how the content of the ASIC registers can be read and written based on a simplified hardware emulation of the ASIC digital subsystem.
