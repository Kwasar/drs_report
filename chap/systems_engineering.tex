\section{Systems Engineering Considerations}
\label{sec:systems_engineering}

\subsection{Power Budget of the Readout Electronics}
\label{sec:power_budget}
This power budget (see tab. \ref{tab:power_budget}) is based on the power consumption in the data sheets of the electronic components used in the design.
The power consumption of the FPGA is a large estimate, since it depends on the implemented logic.
Also the DRS's power consumption is an estimate based on the value given in a book\footnote{Typical specifications for pixel chips used in particle physics: Power dissipation per pixel \textless $50\mu m$.\cite{rossi2006pixel}}, the real value will depend on the final design.
\begin{table}[H]
	\centering
    \includegraphics[width=1\textwidth]{power_budget.jpg}
    \caption[Power Budget]{Power budget of the DRS electronics.}
	\label{tab:power_budget}
\end{table}

The budget seems to be realistic, since the total consumed power is close to the 0.9W\cite[p. 11, tab. 4]{tantalumproject2016} of the RADEM mission, which uses a similar DRS.

\subsection{Data Rate and digital resources}
Based on the timing behavior of the FPGA hardware (see fig. \ref{fig:Simulation}) as well as the operating clock frequencies of the interfaces, the maximum data rate generated by the directional radiation sensor through the readout electronics can be estimated. 

\begin{table}[H]
	\centering
    \includegraphics[width=1\textwidth]{Data_budget.JPG}
    \caption[Data Budget]{Maximum data rate and data volume of FPGA sensor readout operation.}
	\label{tab:data_budget}
\end{table}

Another important consideration are the available hardware gate resources on the FPGA. In the Libero SoC software, the required digital resources can be estimated.

\begin{table}[H]
	\centering
    \includegraphics[width=0.2\textwidth]{resource.JPG}
    \caption[Components]{Share of hardware components used for the synthesis of the design on the SmartFusion2 M2S050}
	\label{tab:resource}
\end{table}

These figures can be compared to the hardware resources of the current choice for the FPGA.

\begin{table}[H]

\caption[]{UART Core specifications \cite{Actel}}
    \label{tab:11}
    
  \begin{center}  
  \begin{tabular}{|r|l|}
  \hline
  \textbf{Part}  & \textbf{ProASIC3 A3P1000} \\ \cline{1-2}
  
  \textbf{System Gates} & 1 Million \\
  \textbf{D-Flip-Flops} & 24576\\ 
  \textbf{RAM Kbits} & 144 \\
  \textbf{FlashROM} & 1K\\
  \textbf{Integrated PLL} & 1 \\
  \textbf{Global Signals} & 18 \\
  \textbf{I/O Banks} & 4 \\
  \textbf{Single-Ended I/Os} & 154 \\
  \textbf{Differential I/O Pairs} & 35 \\
  \hline
  
\end{tabular}
\end{center}
\end{table}

If the DFF (D-Flip-Flop) count between the existing design and the data in the table for the ProASIC3, it becomes evident, that there is a passable margin between the required and available gates on the FPGA.


\subsection{Radiation tolerance of the ProASIC3 A3P1000}

Low power flash-based FPGAs present a reliable component for system integration on a spacecraft. Its main advantage is that configuration memories are not volatile and hence do not require SRAM based volatile memory to reload the device configuration. Furthermore, they are a lot more resistant against radiation induced single-event upsets (SEU). In these cases one of the bits in the memory flips and it can ultimately make memory unreadable or change the configuration in a way that the hardware design becomes unusable. 
The ProASIC3 comes with an entire set of configuration bits that all boast a triple module redundancy. 
\newline

However despite all, the flash based configuration memory as well as the floating gate switches and CMOS logic are still susceptible to the degrading effect of the total ionizing dose (TID) and SEU.
\newline

The ProASIC3 has been tested for its radiation tolerance.

\begin{table}[H]
	\centering
    \includegraphics[width=0.8\textwidth]{Radiation.JPG}
    \caption[]{TID Effects from Proton Irradiation}
	\label{tab:radiation}
\end{table}



\subsection{\texorpdfstring{$I^2C$}{TEXT} Interfaces}
\label{sec:i2c_interfaces}
Some electric components in the design can only be accessed via an $I^2C$ interface. 
Their addresses are defined in the following table:
\begin{table}[H]
	\centering
    \includegraphics[width=0.5\textwidth]{i2c_interfaces.jpg}
    \caption[$I^2C$ Interfaces]{$I^2C$ interface addresses.}
	\label{tab:i2c_interfaces}
\end{table}
